\documentclass{tufte-handout}

%\geometry{showframe}% for debugging purposes -- displays the margins

\usepackage{amsmath}

% Set up the images/graphics package
\usepackage{graphicx}
\setkeys{Gin}{width=\linewidth,totalheight=\textheight,keepaspectratio}
\graphicspath{{graphics/}}

\title{Introduction to Lipids}
\author{}
\date{}  % if the \date{} command is left out, the current date will be used

% The following package makes prettier tables.  We're all about the bling!
\usepackage{booktabs}

% The units package provides nice, non-stacked fractions and better spacing
% for units.
\usepackage{units}

% The fancyvrb package lets us customize the formatting of verbatim
% environments.  We use a slightly smaller font.
\usepackage{fancyvrb}
\fvset{fontsize=\normalsize}

% Small sections of multiple columns
\usepackage{multicol}

% Provides paragraphs of dummy text
\usepackage{lipsum}

% These commands are used to pretty-print LaTeX commands
\newcommand{\doccmd}[1]{\texttt{\textbackslash#1}}% command name -- adds backslash automatically
\newcommand{\docopt}[1]{\ensuremath{\langle}\textrm{\textit{#1}}\ensuremath{\rangle}}% optional command argument
\newcommand{\docarg}[1]{\textrm{\textit{#1}}}% (required) command argument
\newenvironment{docspec}{\begin{quote}\noindent}{\end{quote}}% command specification environment
\newcommand{\docenv}[1]{\textsf{#1}}% environment name
\newcommand{\docpkg}[1]{\texttt{#1}}% package name
\newcommand{\doccls}[1]{\texttt{#1}}% document class name
\newcommand{\docclsopt}[1]{\texttt{#1}}% document class option name

\begin{document}

\maketitle% this prints the handout title, author, and date

\begin{abstract}
\noindent This unit will cover lipid metabolism, with lectures on structure and properties, digestion, synthesis, oxidation and transportation.  This particular lecture will cover the general properties of lipids, including fatty acid, steroid and tri- and diglycerides.  For more details on general fatty acid properties refer to Chapter 30 in Lippincott's Illustrated Reviews in Biochemistry available in reserve\cite{Ferrier2017}.
\end{abstract}

\tableofcontents

\pagebreak
\section{Learning Objectives}

\begin{itemize}
\item Understand the different roles of lipids in our bodies
\item Describe the structure and functions of triacylglycerols (triglycerides)
\item Recognize that phospholipids are amphipathic and play an important role as structural components within our body
\item Identify the structure and functions of cholesterol and other steroids
\item Use the common, n- and $\omega$ nomenclature systems to describe fatty acids, and be able to draw fatty acids based on these various naming systems
\item Describe the structure of fatty acids and analyze how this affects their packing, solubility and physical state
\item Explain the roles of the essential fatty acids, including what makes them essential

\end{itemize}

\section{Function of Lipids}

\subsection{Structural Roles of Lipids}

\newthought{Phospholipids are a subclass of diacylglycerides important for cellular membranes.}
\subsection{Roles in Energy Storage}

\section{Classes of Lipids}

\subsection{Cholesterol and other Sterols}

\subsection{Glycerolipids}

\section{Properties and Structures of Fatty Acids}

\subsection{Classes of Fatty Acids}

The acyl chains that are conjugated to glycerol\sidenote{In the case of triglycerides}, a phospholipid head group\sidenote{In the case of diacylglycerides or phospholipids} or steroids\sidenote{In the case of esterified cholesterol, for example} are defined by two aspects of their structure.  The first is their length, or the number of carbon atoms in the fatty acid.  Based on this criteria, fatty acids are grouped together as short, medium, long or very long-chain fatty acids (see Table \ref{tab:fa-length}).  Shorter fatty acids are more soluble, but contain less energy (since energy is released when each bond is broken).  

\newthought{Saturation levels is another criteria for comparing fatty acids.}  While saturated fatty acids have no double bonds, both monounsaturated\sidenote{containing one double bond} and polyunsaturated\sidenote{containing more than one bond} fatty acids can be made.  These double bonds are generated by a class of enzymes known as \emph{desaturases}.  For example Stearoyl-CoA desaturase\sidenote{also known as $\Delta$-9-desaturase} can introduce a double bond at the $\Delta$-9 position\sidenote{more about what this means in the nomenclature section.  Lots of footnotes today!} of a fatty acid, so could convert a saturated fatty acid into a monounsaturated fatty acid.

\newthought{The type of double bond is a third criteria.}  In nature most bonds are in what we refer to as the \textit{cis} position.  This means that the hydrogens on either side of a double bond are on the same side.  The opposite stereoisomer, where the hydrogens are on opposite sides of the double bond are known as \textit{trans} fatty acids, or more commonly as trans fats.  While these are rare in nature, they became quite abundant during the industrial process of converting unsaturated fats (such as those in corn or canola oil) to saturated fats.  This hydrogenation made most double bonds into single bonds, but sometimes also switched the stereoisomer orientation from a \textit{cis} to a \textit{trans} orientation.  Nutritional epidemiology studies associated trans fatty acid intake with about a 50\% increased risk of coronary heart disease \citep{Willett1993}.  Since these are dispensible to the human diet, and because of their health risks, trans fats are limited or banned in most countries, including the United States which plans to have them removed from the food supply by next year.  

\subsection{Fatty Acid Nomenclature Systems}

Based on the above criteria (length, location and types of double bonds) there is a wide variety of potential fatty acids.  As such three naming systems have been used, their common names, the $\Delta$ notation and the $\omega$ notation.  These are interchangeable and provide generally redundant information about a fatty acid.

\newthought{The common name} is generally the hardest to remember.  Each fatty acid is given a different name, like stearic acid, oleic acid, or $\alpha$-linoleic acid.  Often these common names are offshoots of foods that these fatty acids were found in.  Without remembering the name to structure comparason it is very hard to guess anything about the physical properties from a common name.

\newthought{The $\Delta$ system} has two parts.  The first part refers to the length, so a C16:0 means a fatty acid that is 16 carbons in length, but with \emph{no} double bonds\sidenote{This fatty acid's common name is palmitate, now is it a saturated fatty acid, or an unsaturated fatty acid?}.  A fatty acid that is C16:1 has one double bond, C16:2 has two and so forth.  This is useful for identifying two of our three criteria length and saturation level, but it does not tell us about the location and isomer of the double bond.  Therefore the $\Delta$ system adds another piece of information, how many atoms from the acidic end the double bond is located at.  Palmitoleic acid is a C16:1$\Delta$9-\textit{cis} fatty acid.  It can be generated by the generation of a double bond at the 9\textsuperscript{th} carbon, starting from the acid end.  This is the product of the enzyme Stearoyl-CoA desaturase acting on palmitic acid, since that enzyme has specificity for generating \textit{cis} bonds at the $\Delta$9 position.

\newthought{The $\omega$ system}, also known as the n-1 system is very similar, but instead counts from the free end, not the acid end.  Going back to our example of Palmitoleic acid, while it is a C16:1$\Delta$9-\textit{cis} fatty acid, it is also a C16:1$\omega$7-\textit{cis} fatty acid.  Count the carbons from one end to the other and convince yourself, of the numbering.

\subsection{Polyunsaturated Fats}

\subsection{Essential $\omega$-3  and $\omega$-6 Fatty Acids}

\newthought{$\omega$-6 and $\omega$-3 fatty acids can compete for many of the same enzymes.}

\bibliography{library}
\bibliographystyle{plainnat}

\end{document}
