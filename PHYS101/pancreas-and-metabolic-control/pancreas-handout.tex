\documentclass{tufte-handout}

%\geometry{showframe}% for debugging purposes -- displays the margins

\usepackage{amsmath}

% Set up the images/graphics package
\usepackage{graphicx}
\setkeys{Gin}{width=\linewidth,totalheight=\textheight,keepaspectratio}
\graphicspath{{graphics/}}

\title{Pancreatic Hormones and Metabolic Control}
\author{Dave Bridges, Ph.D.}
%\date{24 January 2009}  % if the \date{} command is left out, the current date will be used

% The following package makes prettier tables.  We're all about the bling!
\usepackage{booktabs}

% The units package provides nice, non-stacked fractions and better spacing
% for units.
\usepackage{units}

% The fancyvrb package lets us customize the formatting of verbatim
% environments.  We use a slightly smaller font.
\usepackage{fancyvrb}
\fvset{fontsize=\normalsize}

% Small sections of multiple columns
\usepackage{multicol}

% Provides paragraphs of dummy text
\usepackage{lipsum}

% These commands are used to pretty-print LaTeX commands
\newcommand{\doccmd}[1]{\texttt{\textbackslash#1}}% command name -- adds backslash automatically
\newcommand{\docopt}[1]{\ensuremath{\langle}\textrm{\textit{#1}}\ensuremath{\rangle}}% optional command argument
\newcommand{\docarg}[1]{\textrm{\textit{#1}}}% (required) command argument
\newenvironment{docspec}{\begin{quote}\noindent}{\end{quote}}% command specification environment
\newcommand{\docenv}[1]{\textsf{#1}}% environment name
\newcommand{\docpkg}[1]{\texttt{#1}}% package name
\newcommand{\doccls}[1]{\texttt{#1}}% document class name
\newcommand{\docclsopt}[1]{\texttt{#1}}% document class option name

\begin{document}

\maketitle% this prints the handout title, author, and date

\begin{abstract}
\noindent This lecture will cover the roles of insulin and glucagon in normal and pathophyiological states, especially diabetes.  These hormones are the major glucose controlling hormones and are both secreted from the pancreas.  This lecture covers the following pages in the textbook: 341-344 \cite{Widmaier2013}.  
\end{abstract}

\tableofcontents

\pagebreak

\section{Learning Objectives}
For this lecture, the learning objectives are:
\begin{itemize}
\item Name the cell types of the Islets of Langerhans and name the hormones secreted by them.
\item Describe the main targets and functions of glucagon.
\item List the major factors that stimulate or inhibit glucagon and insulin.
\item Describe the important physiological roles of insulin.
\item List the major actions of insulin in muscle, adipose tissue, and liver.
\item Explain briefly the mechanism of glucose uptake into the muscle.
\item Name the tissues in which insulin facilitates glucose uptake and those in which insulin does not facilitate glucose uptake.
\item List the major factors that stimulate or inhibit insulin secretion.
\item Draw an oral glucose tolerance test (oGTT)  (glucose, insulin, and glucagon levels) and explain the conditions and describe what is occurring and why.  Explain how the two hormones act to promote glucose homeostasis in the plasma and in the key target tissues for each of these hormones.
\item State which nutrient storages are preferably used for short-term regulation of energy metabolism if no nutrients are available from the GI tract.
\item Discuss the hormones involved, fuel storage capacity, fuel storage consumption, and glucose (or fatty acid) levels during 1) the postprandial period, 2) the post-absorptive period, 3) fasting.
\item List the insulin-counteracting hormones and their roles in glucose homeostasis. Discuss the hormones involved in minute-to-minute regulation and long-term regulation of glucose homeostasis.

\end{itemize}

\section{Sites of Glucose Regulation}

\newthought{Glucose is maintained in a very narrow range}, between 4.4 to 6.1 mmol/L.  These levels need to be re-established after changes in feeding status, or energy utilization.  In general, when glucose levels decrease, glucagon is released from alpha cells of the pancreas to promote glucose production, either from glycogen breakdown or gluconeogenesis.  Alternately, after a meal when glucose levels increase, insulin is secreted from beta cells of the pancreas causing glucose levels to decrease.

For the purposes of the acute maintenance of glucose homeostasis, four organs are the most important; the pancreas, liver, muscle and adipose tissue.  The pancreas senses changes in glucose levels and responds by releasing either glucagon or insulin. 

\newthought{Glucose is taken up down a concentration gradient} from the blood into most tissues including liver, pancreas, kidneys and the brain.  However, for glucose to enter into muscle and fat tissue, insulin is required.  This is accomplished by moving GLUT4 transporters from intracellular storage sites to the plasma membrane, allowing for glucose influx.  It is important to remember that this is \emph{passive} transport, as glucose still needs to go down a concentration gradient, and no ATP or concentration gradient is used to facilitate the entry of glucose.

\newthought{Glucose can be stored in esterified form as glycogen\sidenote{This is known as glycogenesis}}.  To form glycogen, glucose must first be converted through glucose-1-phosphate into UDP-glucose.  This activated form of glucose is then added onto existing glycogen chains through the activity of an enzyme named glycogen synthase.  In addition to being regulated by protein phosphorylation and sub-cellular location, glycogen synthase is also allosterically activated by glucose-6-phosphate, promoting increased glycogen synthesis when glucose levels in the cell are high.

\newthought{To liberate glucose from stored glycogen, an enzyme known as glycogen phosphorylase is activated}.  This enzyme hydrolyses glycogen, releasing glucose-1-phosphate, which can then be dephophorylated into glucose for glycolysis or release into the blood stream.  This is the preferred source of short term glucose maintenance.  In addition to post-translational modifications and recruitment to the glycogen pellet by accessory proteins, glycogen phosphorylase is allosterically activated by energy stress such as increases in AMP, or negatively by increases in glucose-6-phosphate levels.

\newthought{Gluconeogenesis is the generation of glucose from non-carbohydrate precursor molecules}.  These typically include amino acids, lactate and the products of fatty acid oxidation.  The vast majority of gluconeogenesis occurs in the liver, and generally is important for glucose production from proteins and lipids after glycogen stores are depleted.  This process is similar to reverse glycolysis though in several cases different enzymes are used.  The rate limiting enzymes in gluconeogenesis are phosphoenolpyruvate carboxykinase, fructose-1,6-bisphosphatase and glucose-6-phosphatase.  These enzymes are under both transcriptional and post-translational control as described below.

\section{Pancreatic Cell Types}

In order to balance the energy requirements of all tissues, blood glucose is primarily controlled via endocrine and neuroendocrine mechanisms.  The primary mediators are insulin and glucagon which are secreted from the pancreas during times of hyper and hypoglycemia respectively.  These two peptide hormones are released from two cell types in the pancreas, the alpha-cells which release glucagon and the beta-cells which release insulin.  Both cell types are located in the Islets of Langerhans within the pancreas.  Delta cells release somatostatin, which reduces both glucagon and insulin secretion.  Finally in the pancreas are a large number of exocrine cells, which secrete enzymes into the stomach for digestion.


\section{Insulin Promotes Glucose Storage}

Insulin was discovered by Frederick Banting and his colleagues at the University of Toronto in 1921.  They performed experiments in which they injected extracts from pancreas fractions into dogs which had their pancreas' surgically removed.  They showed that a secreted substance from the pancreas lowered blood glucose in these dogs \cite{Banting1922}.  They were then able to confirm that this treatment was also effective in children with diabetes \cite{Banting1922a}.  This work led to Banting and John Macleod winning the Nobel Prize in Medicine and Physiology in 1923.

When glucose levels are raised, such as after a meal, insulin has four main functions, all of which aim to reduce blood glucose levels:

\begin{enumerate}
\item Promotes the uptake of glucose from the blood into muscle and adipose tissue.  
\item Enhances the synthesis of glycogen and triglycerides in liver, adipose and muscle.  
\item Inhibits gluconeogenesis, or the production of glucose from non-glucose precursors such as amino acids and lipids.
\item Promote the breakdown of glucose via glycolysis.
\end{enumerate}

\subsection{Insulin Release and Insulin Signal Transduction}

Beta cells in the pancreas generate insulin and store it in pre-formed secretory granules.  After the depolarization of beta cells in response to high levels of glucose, these secretory granules are exocytosed and their contentes are released into the blood.

Insulin functions by binding to and activating a receptor tyrosine kinase.  This receptor transautophosphorylates itself generating binding sites for phosphotyrosine binding proteins known as insulin receptor substrates.

These proteins are also phosphorylated by the insulin receptor, which creates binding sites for a phosphatidylinositol-3-kinase (PI3K).  This kinases generates the key second messenger in insulin signaling, phosphatidylinositol-(3,4,5)-triphosphate (PIP$_3$).  Most known functions of insulin are blocked when PI3K is inhibited \cite{Kanai1993}.

Once PIP$_3$ is generated by insulin stimulation, it can diffuse along the internal membranes of the cells.  This lipid second messenger recruits two important protein kinases, Akt (also referred to as PKB) and PDK1.  Both of these proteins have domains called pleckstrin homology domains which recruit the kinases together to the plasma membrane.  Once there, PDK1 and another protein kinase called mTORC2 are able to phosphorylate and activate Akt.  Once activated, \emph{Akt is the most important protein kinase in mediating insulin function}.

\subsection{Regulation of Glucose Uptake in Muscle and Adipose Tissue}

In fat and muscle tissue, insulin promotes the movement of a facilitative glucose transporter named GLUT4.  Normally GLUT4 resides in intracellular compartments, but in response to insulin vesicles from these compartments fuse with the plasma membrane, inserting GLUT4 into the extracellular membrane.  This allows for glucose to enter fat and muscle cells.

In both fat and muscle, the PI3K/Akt dependent signaling pathways are absolutely required for insulin stimulated glucose uptake.  The major targets of Akt in this signaling pathway are AS160 and RGC1/2, two proteins which regulate the activity of small GTPases involved in GLUT4 translocation.  The full mechanisms regulating GLUT4 trafficking are not yet fully understood.

\subsection{Regulation of Lipid and Glycogen Synthesis in Muscle, Liver and Adipose Tissue}

Insulin stimulated glucose uptake\sidenote{in muscle and fat, but not liver} provides glucose as a substrate for making fat and glycogen.  In addition to the increases in substrates, insulin will also allosterically activate the enzyme by generating large amounts of glucose-6-phosphate which can activate glycogen synthase.  This is augmented by post-translational regulation of glycogenic enzymes by insulin.

\newthought{Both glycogen synthase and glycogen phosphorylase are regulated by protein phosphorylation}, in an Akt-dependent manner.  In the case of glycogen synthase, the phosphorylated form is relatively inactive, and is resistant to allosteric activation by glucose-6-phosphate \cite{Friedman1963}.  Glycogen synthase is phosphorylated by several protein kinases including AMPK and GSK-3 \cite{Parker1982}.  In addition to inactivating the upstream kinases, insulin also activates a protein phosphatase, which removes the phosphate groups.  In a co-ordinated way, protein phosphorylation activates glycogen phosphorylase \cite{Krebs1964}.  This means that when these enzymes are phosphorylated, the balance tips towards glycogenolysis, and when they are dephosphorylated glycogen is synthesized.

\subsection{Regulation of Gluconeogenesis in the Liver}

The activation of glucose uptake and glycolysis leads to increased levels of several glycolytic intermediates which themselves will reduce gluconeogenesis.  The most important of these is Fructose-2,6-bisphosphate which is raised during glycolysis and inhibits FBPase, one of the key rate limiting steps in gluconeogenesis.

In addition to these effects, both G6Pase and PEPCK, two other rate limiting enzymes are regulated transcriptionally.  Akt phosphorylates and inactivates the transcription factor FOXO which would normally drive the expression of these enzymes.  Therefore when insulin activate the PI3K/Akt cascade, FOXO mediated transcription of G6Pase and PEPCK is decreased and the levels of these enzymes are reduced, decreasing gluconeogenesis.

\section{Glucagon Promotes Glucose Elevation}

 When glucose levels are low, glucagon is released from alpha cells in the pancreas.  This promotes the breakdown of glycogen stores in liver and muscle, and the generation of glucose from gluconeogenic precursors.  Glucagon receptors exist mainly in the liver, so glucagon does not exert its main catabolic effects on either adipose or muscle tissue\sidenote{Think about how this affects how glucagon might have similar, or different effects than adrenaline}. 

The mechanisms which underlie hypoglycemia induced glucagon release are incompletely understood.  What is clear however, is that when blood glucose levels decrease, glucagon is released from the alpha cells of the pancreas into the portal vein.

\subsection{Glucagon Signal Transduction}

Adrenergic-receptor coupled mediated cAMP synthesis was the first example of a hormonal second messenger.  Earl Sutherland was interested in the regulation of glycogenolysis and he noticed that if he added adrenaline to intact cells, he could accelerate glycogen breakdown, but if he added it to lysed cells he could not.  In his key experiment he treated one set of livers with adrenaline, then lysed them.  He then added that lysate to a second set of livers which had already been broken up.  He found that there was an internal factor (later identified as cAMP) in the stimulated tissues, that could accelerate glycogenolysis in the other tissues \cite{Rall1956}.  For this work, Sutherland won the Nobel Prize in Medicine and Physiology in 1971.

In metabolism, the main effector of cAMP in cells is Protein Kinase A (PKA).  This protein kinase is allosterically activated by cAMP and phosphorylates a wide variety of important metabolic substrates.  The identification of PKA and its role in carbohydrate homeostasis led to Fisher and Krebs winning the Nobel Prize in Medicine and Physiology in 1992.  The major role of glucagon is to stimulate glucose release, both by mobilizing glycogen stores and inducing gluconeogenesis.  The mechanisms for this are identical to those for adrenaline, as both of these hormones activate Gs-linked receptors and result in PKA activation in the liver.

\subsection{The Primary Target of Glucagon is the Liver}

As described above, glucagon stimulates the breakdown of glycogen.  This proceeds via protein phosphorylation of both glycogen phosphorylase (which activates this enzyme) and glycogen synthase (which inactivates that enzyme).  In combination, this leads to a breakdown of glycogen into glucose.  This is opposite to the post-translational effects of insulin.

PKA is the primary mediator of the activation of glycogen phosphorylase.  Once activated by adrenergic signaling, PKA phosphorylates and activates glycogen phosphorylase kinase.  This kinase in turn, phosphorylates and activates glycogen phosphorylase\cite{Krebs1956}.  PKA also directly phosphorylates glycogen synthase, which in concert with the activation of the other glycogen synthase kinases (notably GSK3 and AMPK) leads to increased phosphorylation and inactivation of glycogen synthase.

In addition to the activation of these protein kinases, there is a reduction of glycogen associated protein phosphatase activity\sidenote{Remember, protein phosphatases remove a phosphate group from a protein, while a protein kinase adds it.}.  As a balance, this leads to more highly phosphorylated and therefore more glycogenolytic activities.

\newthought{Glucagon promotes gluconeogenesis.}  In addition to the decreased flux of glycolytic intermediates which allosterically activate gluconeogenesis, there are both post-translational and transcriptional mechanisms by which adrenergic signaling promotes gluconeogenesis.  

Post-translationally, the best studied route by which PKA activates gluconeogenesis is through inactivation of phosphofructokinase-2.  PFK-2 normally generates the carbohydrate Fructose-2,6,-bisphosphate which is a positive regulator of glycolysis and a negative regulator of gluconeogenesis.  The alleviation of this inhibition allows for promotion of the gluconeogenic metabolism.  

Transcriptionally, the transcription factor CREB is phosphorylated by PKA where it plays a role in transcriptionally activating the rate limiting gluconeogenic enzymes PEPCK, FPBase and G6Pase.  These pro-gluconeogenic effects in the liver, mediated by a Gs/cAMP/PKA pathway are identical between glucagon and adrenaline.

\section{Other Glucoregulatory Hormones}

Since glucagon works primarily on liver tissue, different hormonal messengers function to stimulate catabolism of lipid in muscle and fat tissue.  The activation of PKA by GPCR and cAMP signaling pathways leads to glycogen breakdown in muscle via similar mechanisms as those in liver but these are generally activated by adrenaline.  Adrenaline also leads to enhanced lipid and glucose oxidation in muscle primarily as an energy source.

In adipose tissue, these pathways induce lipolysis, via phosphorylation and activation of Hormone Sensitive Lipase (HSL), Perilipin and Adipocyte Triglyceride Lipase (ATGL).  These proteins function to mobilize triglycerides into free fatty acids for use in other tissues, especially muscle.  For more information on the regulation of lipolysis, see \cite{Young2013}.  At an acute level, these do not contribute much to glucose homeostasis.

\newthought{Longer term glucose control is regulated by two other hormones previously discussed, growth hormone and cortisol.}  These hormones are elevated during times of growth or stress where it is important to keep circulating glucose available for other functions.  During a prolonged fast, both GH and cortisol can be released, causing longer-lasting changes which ensure adequate blood supply to the brain.

\section{Pathophysiology Related to Glucose Control}

\subsection{Type I Diabetes Mellitus}

Type I Diabetes is typically caused by autoimmune destruction of pancreatic beta cells.  Without these cells, the pancreas is unable to produce insulin and without careful monitoring and exogenous insulin, blood glucose levels will rise.

\subsection{Insulin Resistance and Type II Diabetes Mellitus}

Type II diabetes occurs as a result of a multi-step process starting with negative feedback loops on insulin signaling.  As more nutrients are stored, for example in obesity metabolic tissues become resistant to the effects of insulin, likely as a way to protect against excessive lipid storage.  Insulin resistance can also be induced by elevated secretion of Growth Hormone (as in Acromegaly) or Cortisol (as in Cushing's Disease).

As tissues become more insulin resistant, more insulin must be secreted by the pancreas to maintain normoglycemia.  If insulin resistance proceeds, more and more insulin will need to be produced and secreted by beta cells.  Eventually the beta cells will be unable to keep up with this demand and glucose levels will rise as the amount of endogenous or exogenous insulin is less and less effective.

\newthought{In the next lecture} we will discuss how appetite is regulated by endocrine factors.

\end{document}
