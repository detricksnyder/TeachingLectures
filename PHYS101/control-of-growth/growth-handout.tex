\documentclass{tufte-handout}

%\geometry{showframe}% for debugging purposes -- displays the margins

\usepackage{amsmath}

% Set up the images/graphics package
\usepackage{graphicx}
\setkeys{Gin}{width=\linewidth,totalheight=\textheight,keepaspectratio}
\graphicspath{{graphics/}}

\title{Endocrine Control of Growth}
\author{Dave Bridges, Ph.D.}
%\date{24 January 2009}  % if the \date{} command is left out, the current date will be used

% The following package makes prettier tables.  We're all about the bling!
\usepackage{booktabs}

% The units package provides nice, non-stacked fractions and better spacing
% for units.
\usepackage{units}

% The fancyvrb package lets us customize the formatting of verbatim
% environments.  We use a slightly smaller font.
\usepackage{fancyvrb}
\fvset{fontsize=\normalsize}

% Small sections of multiple columns
\usepackage{multicol}

% Provides paragraphs of dummy text
\usepackage{lipsum}

% These commands are used to pretty-print LaTeX commands
\newcommand{\doccmd}[1]{\texttt{\textbackslash#1}}% command name -- adds backslash automatically
\newcommand{\docopt}[1]{\ensuremath{\langle}\textrm{\textit{#1}}\ensuremath{\rangle}}% optional command argument
\newcommand{\docarg}[1]{\textrm{\textit{#1}}}% (required) command argument
\newenvironment{docspec}{\begin{quote}\noindent}{\end{quote}}% command specification environment
\newcommand{\docenv}[1]{\textsf{#1}}% environment name
\newcommand{\docpkg}[1]{\texttt{#1}}% package name
\newcommand{\doccls}[1]{\texttt{#1}}% document class name
\newcommand{\docclsopt}[1]{\texttt{#1}}% document class option name

\begin{document}

\maketitle% this prints the handout title, author, and date

\begin{abstract}
\noindent This lecture covers endocrine control of growth.  The primary hormone that mediates growth is, unsurprisingly known as growth hormone\sidenote{sometimes refered to as somatorop(h)in, hGH, or when generated recombinantly rhGH}.  This lecture covers the following pages in the textbook: 350-353 and 358-359 \cite{Widmaier2013}.
\end{abstract}

\tableofcontents

\pagebreak

\section{Learning Objectives}
For this lecture, the learning objectives are:
\begin{itemize}
\item List the hormones important for growth at key times in a person's life.
\item Describe the functions of human growth hormone on growth (bones and soft tissues), and on metabolism, and the regulation of its secretion.  Explain what 'rhGH' means.
\item State the "dual effector hypothesis" for GH actions, and the relative roles of GH and IGF-1 in growth control. 
\item Describe the interactions among all the key growth-regulating hormones at key times of a person's life: in utero, neonatally, childhood, puberty, adulthood, and senescence.
\item Describe the daily regulation of GH levels and the physiological relevance of these cycles.

\end{itemize}

There are several hormones that are involved in normal growth.  The most important is growth hormone, but insulin, thyroid hormones, Vitamin D and sex hormones are also very important.  These are covered in separate lectures.  Generally proper growth (length and mass increase) requires proper nutrition\sidenote{both macro- and micronutrients} and a good psychosocial environment.   

\newthought{Humans undergo two major growth phases.}  During the first two years there is a dramatic increase in bone, muscle and other organ size.  The second major growth phase, which occurs during puberty is at 12-20 years old.  Sex hormones\sidenote{estrogen and testosterone} cause this growth spurt by increasing the levels of both GH and IGF-1.

\section{Regulation of Growth Hormone and IGF-1 Levels}

The main regulator of growth are GH and IGF-1, which is regulated by GH.  These hormones are under control of the hypothalamus and pituitary.

\subsection{Hypothalamic and Pituitary Control of Growth Hormone}

Growth hormone is released from the somatotroph cells in the anterior pituitary.  These cells secrete growth hormone into the circulation upon PKA activation.  The two primary regulators of GH secretion are the two hypothalamic hormones GHRH\sidenote{growth hormone releasing hormone} and somatostatin which are secreted into the hypophyseal portal system.  The GPCR receptors of these hormones are Gs and Gi linked receptors respectively, so either promote or inhibit the activation of PKA.

\newthought{Growth hormone is highest during youth while people are actively growing.}  As a person ages, the amount of growth hormone decreases.  Growth hormone also undergoes a normal diurnal rhythm.  GH levels are highest shortly after going to sleep and lower during the day.  Because of this, most growth occurs during sleeping when nutrients can be used for that purpose and are not needed for normal activities.

\subsection{Growth Hormone Regulates IGF-1 Secretion}

While growth hormone has some effects on growth, several important effects are mediated by another hormone IGF-1\sidenote{insulin-like growth factor 1}, whose expression is regulated by growth hormone.  IGF-1 is a protein hormone synthesized and released primarily from the liver and signals through receptor tyrosone kinases.  In addition to regulation by growth hormone, IGF-1 can also be regulated by nutrient status, such that if a person is starving, IGF-1 levels are low, even if GH levels are elevated.


\section{Effects of Growth Hormone}

The main role of the GH/IGF-1 axis is to encourage muscle, bone and other organ growth and to divert important nutrients such as proteins, carbohydrates and lipids towards that end.  As such, GH/IGF-1 has catabolic actions in storage tissues such as adipose, but anabolic actions in growing bones and muscles.

\subsection{Growth Hormone and IGF-1 Signaling}

GH functions through a receptor\sidenote{The Growth Hormone Receptor.} of the JAK/STAT family.  These receptors function primarily by phosphorylating a class of transcription factors known as STATs\sidenote{Signal Transducer and Activator of Transcription} which then produce new mRNAs.  IGF-1 on the other hand functions through a receptor tyrosine kinase, similar to insulin.  This can result in rapid activation of enzymes that are important for distributing nutrients towards growing tissues.

\subsection{Bone and Soft Tissue Growth}

Bones grow via expansion of a region known as the epiphysial growth plate.  A specialized cell type known as osteoblasts form bone at the edge of the plate in concert with chondrocytes, which lay down cartilage in the interior.

\newthought{Muscle growth occurs primarily via two mechanisms, the induction of protein synthesis and the generation of new muscle cells\sidenote{This is known as myogenesis.}}  Both of these processes are primarily controlled by IGF-1, which activates a nutrient sensing protein kinase called mTORC1\sidenote{mechanistic target of rapamycin}.  The signaling pathway by which IGF-1 activates mTORC1 is very similar to the insulin signaling pathway we will discuss in the next lecture.

\subsection{Regulation of Metabolism}

In addition to promoting protein synthesis in muscle, GH/IGF-1 signaling has several other important peripheral effects.  In order to provide substrates for bone and soft tissue growth, GH induces liver gluconeogenesis and promotes lipid breakdown in adipose tissue.  In addition to these effects, IGF-1 stimulates glucose uptake in muscle as well as promotes glycogen and lipid storage in these tissues.

\section{Integration of GH with Other Endocrine Factors}

Several other hormones work with, or against GH in the control of growth.  As stated above, nutrient status is extremely important for growth hormone signaling but insulin and thyroid hormones also enhance GH signaling.

\newthought{In addition to the direct effects of T$_3$ on bone growth}, thryroid hormone\sidenote{This will be covered by Dr. Parthasarathi in his lectures on the thyroid gland} is absolutely required for growth hormone synthesis in the pituitary.  This occurs via direct activation of GH transcription in the pituitary somatotropes.

\newthought{In contrast to the helpful effects of the sex hormones, thyroid hormones and insulin, cortisol antagonizes many of the effects of GH/IGF-1 signaling}.  In times of stress, when cortisol is elevated nutrients are required for essential functions and growth is interrupted.  

\section{Pathologies Associated with Growth Hormone Signaling}

While growth hormone decreases dramatically with age, the effects of GH on bone growth and muscle development also decrease.  This, while part of the normal aging process is one of the reasons that the elderly are at higher risk for sarcopenia\sidenote{degenerative muscle loss} and fractures.  In addition to these normal aging effects there are two diseases that are associated with altered GH signaling.  This relationship is complex though, as cortisol blocks the release of GHRH in the hypothalamus but also promotes GH synthesis.  This is so that after the stress (and cortisol levels) are resolved, sufficient GH is available to resume normal function.

\subsection{Acromegaly}

A pituitary tumor of the somatotroph cells results in constitutive release of GH into the blood stream, a condition known as acromegaly.  These patients have excessive growth, characterized by increased height, protruding jaw and increased muscle mass.  These patients also tend to be lean\sidenote{due to the constitutive activation of lipolysis} and are insulin resistant\sidenote{potentially due to the disruptive effects of increased fatty acid flux into muscle tissue}.  Acromegalics are often treated with somatostatin, or via surgical removal of the pituitary tumor.

\subsection{Dwarfism}

The other end of the spectrum is dwarfism.  While most cases of dwarfism are due to the activation of FGFR3\sidenote{This is a negative regulator of growth hormone signaling in bone that we have not discussed.  This condition is known as achonrdroplasia.}\cite{Shiang1994}.  A smaller proportion of dwarfism is due to growth hormone deficiency.  These patients are smaller, and have muscle weakness, but also have enhanced risk of hypoglycemia\sidenote{due to impaired fasting induced GH secretion, discussed in the next lecture.}.  This can be caused by several things, including immune destruction of pituitary somatotropes, congenital mutations in the growth hormone or its receptor or in response to chemotherapy or radiation exposure. 

\newthought{As described above, thyroid hormone is an important positive regulator of GH signaling.}  As such, hypothyroidism also resembles growth hormone deficiency as growth hormone signaling is impaired.

In the next lecture we will discuss the role of the pancreas in the regulation of nutrient homeostasis and discuss hormonal control of feeding.
\listoffigures
\listoftables

\bibliography{library}
\bibliographystyle{plainnat}



\end{document}
